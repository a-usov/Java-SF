\documentclass[runningheads]{llncs}
%%%%%%%%%%%%%%%%%%%%%%%%
%PACKAGES
%%%%%%%%%%%%%%%%%%%%%%%%
%\usepackage[utf8]{inputenc}
\usepackage[T1]{fontenc}
\usepackage[english]{babel}

\usepackage{amsmath}
\usepackage{amsfonts}
\usepackage{amssymb}
%\usepackage{marvosym}
%\usepackage{txfonts}
%\usepackage{multicol}
\usepackage{proof-dashed}
\usepackage{graphicx}
\usepackage{listings}
\usepackage{hyperref}
\usepackage{stmaryrd}

\renewcommand{\theequation}{\arabic{section}.\arabic{equation}}

% Algorithm typesetting package
\usepackage[plain, shortend, linesnumberedhidden, ruled]{algorithm2e}
\SetAlFnt{\sffamily \small}
\SetKwRepeat{Do}{do}{while}
\SetKwProg{Fn}{Function}{:}{end}
\SetKwInOut{Input}{Input}\SetKwInOut{Output}{Output}

%% For section renaming in autoref
\addto\extrasenglish{%
  \renewcommand{\sectionautorefname}{\S}%
  \renewcommand{\subsectionautorefname}{\S}%
  \renewcommand{\algorithmautorefname}{Algorithm}%
  %\renewcommand{\exampleautorefname}{Ex.}%
}

\makeatletter
\def\verbatim@font{\linespread{0.75}\normalfont\ttfamily}
\makeatother

% Non standard
%\usepackage{varacca}
\input semmacros

\begin{document}

\title{SFJ: An implementation of Semantic Featherweight Java
    \thanks{Supported by the UK EPSRC grant EP/K034413/1, ``From Data Types to Session Types: A Basis for Concurrency and Distribution'' (ABCD), and by the EU HORIZON 2020 MSCA RISE project 778233
    ``Behavioural Application Program Interfaces'' (BehAPI).}
}
\titlerunning{SFJ}

\author{Artem Usov \and Ornela Dardha\orcidID{0000-0001-9927-7875}}
\authorrunning{A. Usov and O. Dardha}

\institute{School of Computing Science, University of Glasgow, Glasgow, United Kingdom \\
\email{2296905U@student.gla.ac.uk, ornela.dardha@glasgow.ac.uk}}

\maketitle

\begin{abstract}    
    There are two approaches to defining subtyping relations: the \emph{syntactic} and the \emph{semantic} approach.
    In semantic subtyping, one defines a model of the language and an interpretation of types as subsets of this model.
    Subtyping is defined as inclusion of subsets denoting types.
    
    An orthogonal subtyping question, typical of object-oriented languages, is the \emph{nominal} versus the \emph{structural} subtyping.
    Dardha \emph{et al.} \cite{Dardha2013,Dardha2017} defined boolean types and semantic subtyping for Featherweight Java (FJ) and integrated both nominal and structural subtyping, thus exploiting the benefits of both approaches.
    However, these benefits were illustrated only at a theoretical level, but not exploited practically.

    We present SFJ---Semantic Featherweight Java, an implementation of FJ which features boolean types and semantic subtyping and integrates nominal as well as structural subtyping.
    The benefits of SFJ, illustrated in the paper and the accompanying video (with audio/subtitles) \cite{UD20}, show how static typechecking of boolean types and semantic subtyping gives higher guarantees of program correctness, more flexibility and compactness of program writing.
    \keywords{Nominal subtyping \and Structural subtyping  \and Semantic Featherweight Java \and Object-oriented languages \and Boolean types \and Type theory.}
\end{abstract}


\section{Introduction}
There are two approaches to defining subtyping relations: the \emph{syntactic} and the \emph{semantic} approach.
Syntactic subtyping \cite{Liskov1994} is more mainstream in programming languages and is defined by means of a set of formal deductive subtyping rules.
Semantic subtyping \cite{Castagna2005,Cas05,Frisch2008} is more recent and less known:  one defines a formal model of the language and an interpretation of types as subsets of this model. Then, subtyping is defined as set inclusion of subsets denoting types.

Orthogonally, for object-oriented languages there are two approaches to defining subtyping relations: the \emph{nominal} and the \emph{structural} approach \cite{Integrating,Useful}.
Nominal subtyping is based on {declarations} by the developer and is \emph{name}-based: ``$A$ is a subtype of $B$ if and only if it is declared to be so, that is if the class $A$ extends (or implements) the class (or interface) $B$''.
Structural subtyping instead is based on the \emph{structure} of a class, its fields and methods: ``a class $A$ is a subtype of a class $B$ if and only if the fields and methods of $A$ are a superset of the fields and methods of $B$, and their types in $A$ are subtypes of the types in $B$''.
Intutively, extending a class with more fields and methods results in a smaller set of inhabitants for that class. For example, the set of objects of a class \emph{Student} is smaller than the set of objects of a class \emph{Person} as each Student is a Person, but not the other way around. Hence, \emph{Student} is a structural subtype of \emph{Person}, even if it is not declared to be so.


Dardha \emph{et al.} \cite{Dardha2013,Dardha2017} define boolean types---based on set-theoretic operations such as \textbf{and}, \textbf{not}, \textbf{or}---and semantic subtyping for \emph{Featherweight Java} (FJ) \cite{featherweight}.
This approach allows for the integration of both nominal and structural subtyping in FJ, bringing in higher guarantees of program correctness, flexibility and compactness in program writing.
Unfortunately, these benefits were only presented at a theoretical level and not exploited practically, due to the lack of an implementation of the language and its types and type system.

In this paper, we present SFJ---\emph{Semantic Featherweight Java} \autoref{sec:design}, an implementation of FJ with boolean types and semantic subtyping.
In SFJ the developer has a larger and more expressive set of types, by using boolean connectives \textbf{and}, \textbf{not}, \textbf{or}, with the expected set-theoretic interpretation. On the other hand, this added expressivity does not add complexity. Rather the opposite is true, as the developer has an easier, more compact and elegant way of programming.
SFJ integrates both structural and nominal subtyping, and the developer can choose which one to use.
Finally, as discussed in Dardha \emph{et al.} \cite[\S 8.4]{Dardha2017}, thanks to semantic subtyping, we can easily encode in SFJ standard programming constructs and features of the full Java language, such as lists, or overloading classes via multimethods \cite{BC97}, which are missing in FJ, thus making SFJ a more complete language closer to Java.

\begin{example}[Polygons]
    \label{polygons}
    This will be our running example both in the paper and in the tool video \cite{UD20} to illustrate the benefits of boolean types and semantic subtyping developed by Dardha \emph{et al.} \cite{Dardha2013,Dardha2017} and implemented as SFJ.

    Consider the set of polygons, such as triangles, squares and rhombuses given by a class hierarchy.
    We want to define a method \emph{diagonal} that takes a polygon and returns the length of its longest diagonal. This method makes sense only if the polygon passed to it has at least four sides, hence triangles are excluded.
    In Java this could be implemented in the following ways:
    $$
        \begin{array}{l}
            \synclass Polygon \ \{ \ldots\}
            \\
            \syndecl{Triangle}{Polygon} {\ldots}
            \\
            \syndecl{Other\_Polygons}{Polygon} {
                \\
                \qquad \mathbf{double}\ diagonal(Other\_Polygons \ shape)\ \{\ldots \}
                \\
                \hspace{3.5cm} \ldots
                \\
            }
            \\
            \syndecl{Square}{Other\_Polygons}{\ldots}
            \\
            \syndecl{Rhombus}{Other\_Polygons} {\ldots}
        \end{array}
    $$
    Or by means of an interface \emph{Diagonal}:
    $$
        \begin{array}{l}
            public \ \syninterface Diagonal\ \{
            \\
            \qquad \mathbf{double}\ diagonal(Polygon \ shape);
            \\
            \}
            \\
            \synclass Polygon \ \{\ldots\}
            \\
            \syndecl{Triangle}{Polygon} {\ldots}
            \\
            \synidecl{Square}{Polygon} \synimpl Diagonal \ \{\ldots\}
            \\
            \synidecl{Rhombus}{Polygon} \synimpl Diagonal \ \{\ldots\}
            \\
            \hspace{4.5cm} \ldots
        \end{array}
    $$

    Now, suppose our class hierarchy is such that \emph{Polygon} is the parent class and all other geometric figures extend \emph{Polygon}, which is how one would naturally define the set of polygons.
    Suppose the class hierarchy is given and is part of legacy code, which cannot be changed. Then again, a natural way to implement this in Java is by defining the method \emph{diagonal} in the class \emph{Polygon} and using an \textbf{instanceof}, for example, inside a \textbf{try-catch} construct. Then, an exception would be thrown at runtime, if the argument passed to the method is a triangle.

    We propose a more elegant solution, by combining boolean types and semantic subtyping, where only static typechecking is required and we implement this in SFJ \cite{UD20}: it is enough to define a method \emph{diagonal} that has an argument of type \emph{Polygon} \textbf{and not} \emph{Triangle}, thus allowing the typechecker to check at compile time the restrictions on types:
    $$
        \begin{array}{l}
            \synclass Polygon \ \{\ldots\}
            \\
            \syndecl{Triangle}{Polygon} {\ldots}
            \\
            \syndecl{Square}{Polygon} {\ldots}
            \\
            \syndecl{Rhombus}{Polygon} {\ldots}
            \\
            \hspace{2.5cm} \ldots
            \\
            \synclass Diagonal \ \{
            \\
            %\qquad Diagonal ( )\   \{\ super();\ \}
            \hspace{2.5cm} \ldots \\
            \qquad \mathbf{double}\ diagonal((\textit{Polygon} \textbf{ and not }  \textit{Triangle}) \ shape)\{\ldots\}
            \\
            \}
        \end{array}
    $$
    We can now call \emph{diagonal} on an argument of type \emph{Polygon}: if the polygon is \textbf{not} a \emph{Triangle}, then the method returns the length of its longest diagonal; otherwise, if the polygon is a triangle, then there will be a type error at compile time.
\end{example}


%The technical developments behind semantic subtyping and its properties are complex, however, they are completely transparent to the developer.
%Due to space limit we will remove most theoretical details from this submission, as they are not necessary to understand the design and implementation of SFJ, which is the focus of this paper. We refer the reader to the relevant work \cite{Dardha2013,Dardha2017} and we will include a more complete background on semantic subtyping in a possible full paper.

\paragraph{Structure of the paper:}
\autoref{sec:background} details the technical developments behind semantic subtyping and its properties.
\autoref{sec:design} details the design and implementation of SFJ.
\autoref{sec:conclusion} concludes the paper.

\section{Background}
\label{sec:background}
The technical developments behind semantic subtyping and its properties are complex, however, they are completely transparent to the programmer. The framework is detailed and proved correct in the relevant work by Dardha \emph{et al.} \cite{Dardha2013,Dardha2017}, and SFJ builds on that framework.

In this section we will briefly detail the types and terms of SFJ.

\subsection{Types}
\label{sec:types}
The syntax of types $\tau$ is defined by the following grammar:
$$
\begin{array}{llll}
\tau & ::= & \alpha \ |\ \mu
& \mbox{{\em Types}}
\\
\alpha & ::=  & \zero \ |\ \btypes \ |\ [\wt{l:\tau}] \ |\ \alpha\ \synwedge\ \alpha \ |\ \synneg\ \alpha
\qquad
& \mbox{{\em Field types} ($\alpha $-types)}
\\
\mu & ::=  & \alpha \to \alpha \ |\ \mu\ \synwedge\ \mu \ |\ \synneg\ \mu
& \mbox{{\em Method types} ($\mu $-types)}
\end{array}
$$

The $\alpha$-types are used to type fields and the $\mu$-types are used to type methods.
Type $\zero$ is the empty type.
Type $\btypes$ denotes the \emph{basic} types, such as integers, booleans, etc.
{Record} types $[\wt{l:\tau}]$, where $\wt l$ is a sequence of disjoint labels, are used to type objects.
Arrow types $\alpha \to \alpha$ are used to type methods.

The boolean connectives $\bf{and}$ and $\bf{not}$ in the $\alpha$-types and $\mu$-types have their expected set-theoretic meanings.
We let $\alpha\ \syndiff\ \alpha'$  denote $\alpha\ \synwedge\ (\synneg\ \alpha')$, and $\alpha\ \synvee\ \alpha'$  denote $\synneg(\synneg\ \alpha\ \synwedge\ (\synneg\ \alpha'))$.

\subsection{Terms}
\label{sec:terms}
The syntax of terms is defined by the following grammar and is based on the standard syntax of terms in FJ \cite{featherweight}:
\begin{align*}
     & \mbox{\textit{Class declaration}}  & L \; ::= \; & \syndecl{C}{C}{\wt{\alpha \ a};\ K; \ \wt{M}\ }                         \\
     & \mbox{\textit{Constructor}}        & K \;::=\;   & C\ (\wt{\alpha\ x})\ \{\ \synsuper(\wt{x});\ \wt{\synthis.a}=\wt{x}; \} \\
     & \mbox{\textit{Method declaration}} & M \; ::= \; & \alpha \ m\ (\alpha \ x)\ \{\ \synret e; \}                             \\
     & \mbox{\textit{Expressions}}        & e \; ::=\;  & x\ |\  c\ |\ e.a\ |\ e.m(e) \ | \ \synnew C(\wt{e})
\end{align*}

We assume an infinite set of names, with some special names:
$\mathit{Object}$ denotes the root class, $\synthis$ denotes the current object and $\synsuper$ denotes the parent object.
We let  $A, B,\ldots$ range over classes; $a, b,\ldots$ over fields; $m, n,\ldots$ over methods and $x, y, z, \ldots$ over variables.

A {program} $(\wt{L}, e)$ is a pair of a sequence of class declarations $\wt L$, giving rise to a class hierarchy as specified by the inheritance relation, and an expression $e$ to be evaluated.
%
A class declaration $L$ specifies the name of the class, the name of the parent class it extends, its typed fields, the constructor $K$ and its method declarations $M$.
The constructor $K$ initialises the fields of the object by assigning values to the fields inherited by the super-class and to the fields declared in the current \textbf{this} class.
%
A method declaration $M$ specifies the signature of the method, namely the return type, the method name and the formal parameter as well as the body of the method.
Notice that in our theoretical development we use unary methods, without loss of generality: tuples of arguments can be modelled by an object that instantiates a ``special'' class containing as fields all the needed arguments.
Expressions $e$ include variables, constants, field accesses, method invocations and object creations.


\paragraph{On multimethods}
Since FJ is a core language, some features of the full Java are removed, such as overloading methods.
In our framework, by leveraging the expressivity of boolean connectives and semantic subtyping, we are able to restore overloading, among other features \cite[\S 8.4]{Dardha2017}.
We can thus model \emph{multimethods}, \cite{BC97}, which according to the authors is ``\emph{very clean and easy to understand [...] it would be the best solution for a brand new language}".
As an example, taken from Dardha \emph{et al.}  \cite{Dardha2013,Dardha2017}, consider the following class declarations:
$$
\begin{array}{ll}
\syndecl{A}{\mathit{Object}} {
&
\qquad
\syndecl{B}{A} {
\\
	%\qquad\quad \ldots
%&
	%\qquad\qquad\quad \ldots
%\\
	\qquad\quad \mathbf{int}\ \mathit{length}\ (\mathbf{string}\ s)\{\ \ldots\ \}\qquad
&
	\qquad\qquad\quad \mathbf{int}\ \mathit{length}\ (\mathbf{int}\ n)\{\ \ldots\ \}
\\
}
&
\qquad }
\end{array}
$$
Method {\it length} has type $\mathbf{string} \rightarrow \mathbf{int}$ in class $A$.
However, because class $B$ extends class $A$,  {\it length} has type $(\mathbf{string} \rightarrow \mathbf{int})\ \synwedge\ (\mathbf{int} \rightarrow \mathbf{int})$ in class $B$,
which can be simplified to $(\mathbf{string}\ \synvee\ \mathbf{int}) \rightarrow \mathbf{int}$.

Following FJ \cite{featherweight}, we rule out ill-formed programs, such as declaring a constructor named $B$ within a class named $A$; or multiple fields or methods having the same name; or fields having the same type as the type of the class they are defined in.

% \subsection{Typing Rules} use the subtyping relation $\leq_{\mathcal B}$ defined semantically as set inclusion of types which are interpreted as subsets of a model, we call it $\mathcal B$ of FJ. Details of model $\mathcal B$ and the subtyping relation $\leq_{\mathcal B}$ are fully presented in \cite{Dardha2013,Dardha2017}.

% Typing judgements for expression are of the form $\Gamma \vdash_{\mathcal B} e : \tau$ and for method declarations are $\Gamma \vdash_{\mathcal B} M$. The same for class declarations and programs.

% Let us assume a sequence of class declarations $\wt{L}$.
% We first determine the (structural) type of every class $C$ in $\wt{L}$.
% To this aim, we have to take into account the inheritance relation specified in the class declarations in $\wt{L}$.
% We write ``$a \in C$'' to mean that there is a field declaration of name $a$ in class $C$ within the hierarchy $\wt L$.
% Similarly, we write ``$a \in C$ with type $\alpha$'' to also specify the declared type $\alpha$. Similar notations also hold for method names $m$.


\section{SFJ: Design and Implementation}
\label{sec:design}
\subsection{Valid type definitions}
Since we want to use types $\tau$ in practice in SFJ, we restrict them to \emph{finite trees} whose leaves are basic types $\btypes$ \autoref{sec:terms} with no cycles.
For example, a recursive type $\alpha = [a : \alpha]$ denotes an infinite program tree \textbf{new} $C$(\textbf{new} $C$($\cdots{}$)), hence we rule it out as it is not inhabitable. Similarly, the types $\alpha = [b: \beta]$, $\beta = [a = \alpha]$ create a cycle and thus would not be inhabitable.
Notice that the above types can be defined in Java, but as well are not inhabitable.

SFJ is implemented using ANTLR \cite{parr2013}. We start by defining the grammar of the language in Extended Backus-Naur Form (EBNF), following \autoref{sec:types} and by using ANTLR, we can automatically generate a parser for SFJ and extend it in order to implement the required checks for our types and type system. Running the parser on an SFJ program returns an abstract syntax tree (AST) of that program. 
%
When visiting the AST, we check if the program is well-formed, following the intuition at the end of \autoref{sec:background}. We mark any classes containing fields typed with only basic types as \emph{resolved} otherwise, as \emph{unresolved}.
Using this information, \autoref{alg:types} checks if the type definitions in a program are valid, namely, if they are finite trees whose leaves are basic types with no cycles.
%
At each iteration of the algorithm we can resolve at least one type, until all the types are resolved, and return $True$. However, if there is at least one type unresolved, that means there is a cycle in the type definition, and we return $False$.

\begin{algorithm}[t]
    \SetAlgoLined
    \caption{Validity Check for Type Definitions}
    \label{alg:types}

    \DontPrintSemicolon
    \Input{$classes$, the set of classes marked $resolved/unresolved$ depending if their fields contain only basic types.}
    \Output{$True$ if all classes are valid type definitions, $False$ otherwise.}
    \BlankLine
    \Begin{
        \Do{$resolutionOccured = true$}{ %\tcc {comment}
            $resolutionOccured \longleftarrow false$ \;
            \For{class that is unresolved in classes}{
                $resolved \longleftarrow true$ \;
                \For{field in class that contains a class type}{
                    \If{type of field is unresolved}{
                        $resolved \longleftarrow false$
                    }

                }
                \;
                \If{$resolved = true$}{
                    $class \longleftarrow resolved$ \;
                    $resolutionOccured \longleftarrow true$ \;
                }
            }
        }
        \;
        \uIf{not all classes are resolved}{
            \KwRet $False$
        }
        \Else{
            \KwRet $True$
        }
    }
\end{algorithm}

\subsection{Building semantic subtyping for SFJ}
If \autoref{alg:types} returns $True$, we can proceed to building semantic subtyping for SFJ.
Leveraging the interpretation of types as sets of values to define the semantic subtyping for FJ \cite{Dardha2013,Dardha2017}, in our implementation of SFJ we keep track of the semantic subtyping relation by defining a map from a type to the set of its subtypes, satisfying the property that the set of values of a subtype is included in the set of values of the type.
We start with basic types and define $Universe$ to be the set of all types.
The mapping for basic types is defined as follows.
\begin{equation}
    \label{eq:initialSubtype}
    \begin{array}{llllll}
        & Double   & = \left\{Double, Float, Int, Short, Byte\right\} &  & Float & = \left\{Float, Short, Byte\right\} \\
        & Long     & =    \left\{Long, Int, Short, Byte\right\}       &  & Int   & = \left\{Int, Short, Byte\right\}   \\
        & Short    & =    \left\{Short, Byte\right\}                  &  & Byte  & = \left\{Byte\right\}               \\
        & Boolean  & =    \left\{Boolean\right\}                      &  & Void  & = \left\{Void\right\}               \\
        & Universe & = \{Double, Float, Long, Int,                                                                     \\
        &          & \quad Short, Byte, Boolean, Void\}
   \end{array}
\end{equation}


Note that \emph{Int} is not a subtype of \emph{Float} as a 32-bit \emph{float} cannot represent the whole set of 32-bit \emph{integer} values accurately and therefore \emph{Int} is not fully set-contained in \emph{Float}, however this is not the case for \emph{Int} and \emph{Double}.
Similarly, \emph{Long} is not a subtype of \emph{Double}.

\autoref{alg:subtyping} builds the semantic subtyping relation for all class types of an SFJ program by calling the function \textit{generateRelation}. Given that classes are valid type definitions by \autoref{alg:types}, we are guaranteed that \autoref{alg:subtyping} will terminate.

%
\begin{algorithm}
    \caption{Semantic Subtyping for SFJ}
    \label{alg:subtyping}

    \DontPrintSemicolon
    \Input{$classes$, the set of classes in an SFJ program for which we have not yet defined a subtyping relation.\\
    $relation$, mapping of types to the set of subtypes. Initially this relation\\ only contains the entries defined in Equation \ref{eq:initialSubtype}. \;
    }
    \Begin{
        \Fn{generateRelation(classes: List<Class>)}{
            $unprocessed: List<Class>\ \longleftarrow []$

            \For{class in classes}{
                \If{addClass(class) = $False$}{
                    $unprocessed.add(class)$
                }
            }
            \If{$untyped \neq []$}{
                $generateRelation(unprocessed)$
            }
        }
        \;
        \Fn{addClass(class: Class) $\rightarrow$ boolean}{
            \For{existing class type in relation}{
                \If{checkSuperSet(class, existingClass) = false}{
                    \KwRet $False$\;
                }
                $checkSuperSet(existingClass, class)$\;
            }
            $relation[class].add(class)$ \;
            $relation[Universe].add(class)$ \;
            \KwRet $True$ \;
        }
        \;
        \Fn{checkSuperSet(class: Class, other: Class) $\rightarrow$ boolean}{
            $flag \longleftarrow True$ \;
            \For{field in class}{
                \If{field contains type not in relation}{
                    \KwRet $False$
                }
                \uIf{other does not contain field}{
                    $flag \longleftarrow False$
                }
                \Else{
                    \If{other.field.types does not fully contain field.types}{
                        $flag \longleftarrow False$
                    }
                }
            }
            
            \For{method in class}{
                \If{method contains type not in relation}{
                    \KwRet $False$
                }
                \uIf{other does not contain method}{
                    $flag \longleftarrow False$
                }
                \Else{
                    \If{other.method.types does not fully contain method.types}{
                        $flag \longleftarrow False$
                    }
                }
            }
            
            \If{$flag = True$}{
                $relation[other].add$
            }
        }
    }
\end{algorithm}
%
\subsection{Polygons: Continued}
Let us illustrate the subtyping algorithm with our \emph{Polygons} \autoref{polygons}.
The algorithm generates the following subtyping relation, which also includes the subtyping relation for basic types defined in the first step.
\begin{flalign*}
    &Polygon = \left\{Polygon, Triangle, Square, Rhombus \right\} \qquad Triangle = \left\{Triangle\right\}\\
    &Square = \left\{Square\right\} \qquad  Rhombus  = \left\{Rhombus\right\} \qquad
    Diagonal = \left\{Diagonal\right\}
\end{flalign*}
Recall method \emph{diagonal} in class \emph{Diagonal}, we can see that the result of the set operation on its parameter type gives the following set of polygons:
\begin{flalign*}
    Polygon \textbf{ and not } Triangle = \left\{Polygon, Square, Rhombus \right\}
\end{flalign*}
If we pass to \emph{diagonal} an argument of type \emph{Triangle}, as it is not contained in the computed set of subtypes, we get a typechecking error at compile time \cite{UD20}.

The subtyping algorithm finds all nominal and structural subtypes.
It finds all nominal subtypes as they inherit all fields and methods of their super class, so are always guaranteed to be a superset.
Similarly, given all pairs of types, the algorithm checks if one is a superset of the other, i.e.,\ they share common fields and methods, and if so, they are related by structural subtyping.
For example, type $empty = []$, will have all classes as structural subtypes.
%The above algorithm will without differentiating, find all nominal and structural subtypes as we always consider all other classes in the program to be potential subtype candidates.

Note that both approaches to subtyping have their drawbacks.
%However, it can be argued that structural subtyping is counterproductive in writing logically correct programs.
Consider two structurally equivalent class/record types $coordinate = [x:int, y:int, z:int]$ and $colour = [x:int, y:int, z:int]$.
While structurally they can be used interchangeably in our type system, their ``meaning'' is completely different.
On the other hand, with nominal subtyping in Java one can define an overridden method to perform the opposite logic to what the super class is expecting.
Therefore, both approaches leave an expectation on the developer to check what they are doing is correct, hence the integration of both in SFJ is the key to overcoming these drawbacks, as one can choose which approach to use for a given task.

Furthermore, we have extended FJ with method types, demonstrated below to show another possible implementation of the \emph{Polygons} \autoref{polygons}.
$$
    \synclass Diagonal \ \{
    \mathbf{int}\ diagonal(\textit{(getDiagonal: void$\rightarrow$int)}  \ shape)\{\ldots\}
    \}
$$

We instead only implement the \emph{getDiagonal} method on polygons that support it. We can then specify to only accept classes that implement the method \emph{getDiagonal}. If we pass an argument of type \emph{Triangle}, it is not in the computed set of subtypes, so we get a typechecking error at compile time.
We do not have to generate a relation for all methods in all classes as this is computationally unnecessary and instead
only generate the set of subtypes on encountering a method type.

%In the implementation of SFJ, we are liberal in the use of fields and method which might not be implemented for all types in the subtyping relation rather than restricting only it to those which we know are shared by all types.
%Improper use of this allows for run-time exceptions however it more closely follows the idea of giving the programmer tools to create more expressive programs at the cost of checking program correctness such as with the use of structural subtypes.

\section{Related Work and Conclusion}
\label{sec:conclusion}
Semantic subtyping approach goes back to more than two decades ago \cite{Aiken,Damm}.
Notable lines of work include:
Hosoya and Pierce \cite{XML1,XML2,XML3} who define XDuce, an XML-oriented language designed specifically to transform XML documents in other XML documents satisfying certain properties.
Castagna \emph{et al.} \cite{gentle,Cas05,FCB08} extend XDuce with first-class functions and arrow types and implement it as $\mathbb{C}$Duce. The starting point of their framework is a higher-order $\lambda-$calculus with pairs and projections.
Muehlboeck and Tate \cite{Muehlboeck2018} define a syntactic framework with boolean connectives which has been implemented in the Ceylon programming language \cite{Ceylon2016}.
Castagna \emph{et al.} \cite{Cpi} define $\mathbb{C}\pi$, a variant of the asynchronous $\pi-$calculus \cite{sangiorgi}, where channel types are augmented with boolean connectives.
Ancona and Lagorio~\cite{AL10} define subtyping for infinite types coinductively by using union and object type constructors, where types are interpreted as sets of value of the language.
Bonsangue \emph{et al.}~\cite{BRABR14} study a coalgebraic approach to coinductive types and define a set-theoretic interpretation of coinductive types with union types.
Pearce \cite{Pearce19} defines semantic subtyping for rewriting rules in the Whiley Rewrite Language and for a flow-typing calculus  \cite{Pearce13}.

To conclude, in this paper we presented the design and implementation of SFJ, an extension of Featherweight Java with boolean types and semantic subtyping, which allow for both structural and nominal subtyping in FJ as well as restore standard Java constructs for example, lists and features for example, overloading which were not present in FJ, making SFJ a more complete language.

%% References
\newpage
\bibliographystyle{splncs04}
\bibliography{paper}
\end{document}
