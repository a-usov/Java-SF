%
%  Macro files
%  Originally from Frisch, Castagna, Benzaken
%  
%

%% Ornela
\newcommand{\abstr}[4]{\lambda_{({#1} , {#2})} {#3}.{#4}}
\newcommand{\semval}[2]{\sem{#1}_{\mathcal #2}}



\newif{\ifSHORT}
\newif{\ifLongVersion}
\newif{\ifWithRecords}
\newif{\ifWiths}

\newcommand{\possiblecut}[1]{{\color{black}#1}}   

\newcommand{\maincomment}[1]{%
  \ifMarginalComments{\mbox{}\\[1mm]$\Longrightarrow$\textsf{#1}\mbox{}\\[1mm]} \else {} \fi}

\newenvironment{mylist}{%
\begin{list}{$\bullet$}{\topsep2pt\parskip0pt\partopsep0pt\itemsep3pt\labelwidth10.5mm\labelsep3pt\leftmargin11mm}}%
{\end{list}}

%
%  unsemplice comando che mi dice la versione
%
\usepackage{calc}\newcounter{tempo}\setcounter{tempo}{\time}
\newcommand{\version}[1]{\mbox{}\\[-\baselineskip]%
   \raisebox{#1in}[0in][0in]{\makebox[\textwidth][c]{\rm\small \today:v.\thetempo}}}
\newcommand{\titlepageheader}[1]{\mbox{}\\[-\baselineskip]%
\raisebox{5.3in}[0in][0in]{\makebox[\textwidth][c]{\rm\small #1}}}
\newcommand{\duce}{$\mathbb{C}$Duce }
\newcommand{\cduce}{$\mathbb{C}$Duce }
\newcommand{\cdoge}{$\mathbb{C}$Doge }
\newcommand{\cpi}{$\mathbb{C}\pi$ }
\newcommand{\cobj}{$\mathbb{C}$Obj }
\newcommand{\sempi}{$\pi_\leq$}   % was semantic-$\pi${}}
\newcommand{\emptyl}{\mathsf{empty}}
%
% Daniele
%
% altri commandi
\newcommand{\wt}[1]{\widetilde{#1}}

\newcommand{\synrnd}{\mathbf{rnd}}
\newcommand{\synclass}{\mathbf{class} \ }
\newcommand{\syninterface}{\mathbf{interface} \ }
\newcommand{\synext}{\ \mathbf{extends }\ }
\newcommand{\synimpl}{\ \mathbf{implements}\ }
\newcommand{\synret}{\mathbf{return}\ }
\newcommand{\synthis}{\mathbf{this}}
\newcommand{\synsuper}{\mathbf{super}}
\newcommand{\synfinal}{\mathbf{final}}
\newcommand{\synnew}{\mathbf{new}\ }
\newcommand{\syndecl}[3]{\synclass #1 \synext #2\ \{#3\}}
\newcommand{\synidecl}[2]{\synclass #1 \synext #2\ }

\newcommand{\mmif}{\mathbf{if}}
\newcommand{\mmthen}{\mathbf{then}}
\newcommand{\mmelse}{\mathbf{else}}
\newcommand{\mmclass}{\mathbf{class}}
\newcommand{\mmnew}{\mathbf{new}}
\newcommand{\mmnull}{\mathbf{null}}
\newcommand{\mmextends}{\mathbf{extends}}
\newcommand{\mmint}{\mathbf{int}}
\newcommand{\mmreal}{\mathbf{real}}
\newcommand{\mmbool}{\mathbf{bool}}
\newcommand{\mmtrue}{\mathbf{true}}
\newcommand{\mmfalse}{\mathbf{false}}
\newcommand{\nname}{\mathbf{name}}
\newcommand{\mmreturn}{\mathbf{return}}
\newcommand{\mminstof}{\mathbf{instanceof}}
\newcommand{\mmtry}{\mathbf{try}}
\newcommand{\mmcatch}{\mathbf{catch}}
\newcommand{\mmthrow}{\mathbf{throw}}
\newcommand{\mmlet}{\mathbf{let}}
\newcommand{\mmin}{\mathbf{in}}
\newcommand{\nominal}{\mathbf{nominal}}

\newcommand{\naming}[1]{(#1)^\nname}
\newcommand{\type}{\mathit{type}}
\newcommand{\mbody}{\mathit{body}}
\newcommand{\judge}[2]{#1 \vdash #2}
\newif\ifmai\maifalse


\newcommand{\height}[1]{\hslash(#1)}%{\textit{height}(#1)}
%\DeclareMathOperator{\uparrowop}{\uparrow}
%\DeclareMathOperator{\downarrowop}{\downarrow}
%
% definition enum environment
%
\newenvironment{enum}{\begin{enumerate}\vspace{-5pt}\topsep0pt\parskip0pt\partopsep0pt\itemsep1pt}{\end{enumerate}\vspace{-5pt}}


%
%
\newcommand{\A}{{\cal A}}
\newcommand{\Norm}{{\cal N}}
\newcommand{\exten}{\mathbb E}
\newcommand{\extenf}{\mathbb{E}_f}
\newcommand{\atoms}{{\mathbb T}}
\newcommand{\myitem}{\mbox{}\\[.5mm]\hspace*{3.5mm}--~~}

\newcommand{\dead}{\textsf{0}}
%\newcommand{\mathbb}[1]{\Bbb{#1}}
%\newcommand{\llbracket}{[\![}
%\newcommand{\rrbracket}{]\!]}

\newcommand{\VAR}{\textit{Var}}
\newcommand{\zero}{\mathbf{0}}
\newcommand{\one}{\mathbf{1}}
%\newcommand{\qed}{\mbox{}\hfill\mbox{$\Box$}}
\newcommand{\bij}{\partial}
\newcommand{\para}{~~|~~}
\newcommand{\hasse}[1]{\xymatrix@R-1em@C-2em{#1}}

\newcommand{\todo}[1]{{\bf \underline{TODO:} #1}}
\newcommand{\ignore}[1]{}
\newcommand{\Is}{\colon\!\!\colon \!\!\!\!= }

\newcommand{\txtvee}{\text{\rm ~or~}}
\newcommand{\txtwedge}{\text{\rm ~and~}}

\newcommand{\dom}{\textit{dom}}

%\newcommand{\naturals}{\mathbb{N}}

% Op�rateurs math�matiques
\renewcommand{\P}{{\cal P}}
\newcommand{\Pf}{{\P_f}}
\newcommand{\compl}[2]{{\complement_{#2}} {#1}}
\newcommand{\egdef}{\stackrel{\textrm{\tiny def}}{=}}
\newcommand{\segdef}{\!\!\!\stackrel{\textrm{\tiny def}}{=}\!\!\!}
\renewcommand{\subset}{\subseteq}
\renewcommand{\emptyset}{\varnothing}

\newcommand{\cX}{{\cal X}}

% Les univers
\newcommand{\ubasic}{{\text{\bf basic}}}
\newcommand{\urec}{{\text{\bf rec}}}
\newcommand{\ufun}{{\text{\bf fun}}}

% Basic types
\newcommand{\btypes}{{\mathbb B}}
\newcommand{\semb}[1]{{\cal B} {\llbracket #1 \rrbracket}}
\newcommand{\Types}{\textbf{Types}}
% Les mod�les
\newcommand{\domaine}{{\cal D}}
%\newcommand{\univ}{{\cal U}}
\newcommand{\domwr}{\domaine_\Omega}
\newcommand{\stdmod}{{\cal U}}
\newcommand{\stdmodwr}{{\cal U}_\Omega}
\newcommand{\sem}[1]{{\llbracket #1 \rrbracket}}
\newcommand{\esem}[1]{{\mathbb E}\left( #1 \right)}
\newcommand{\esemd}[1]{{\mathbb E}$D$}
\newcommand{\esemp}[1]{{\cal E}\llparenthesis #1 \rrparenthesis}
\newcommand{\tsem}[1]{\llparenthesis #1 \rrparenthesis}
%\newcommand{\semv}[1]{{\mathbb V}{\llbracket #1 \rrbracket}}


% Les types
\newcommand{\functor}{{\mathbb T}}
\newcommand{\syntypes}{{\mathcal{T}}}
\newcommand{\atomtypes}{\textit{A\/}}%{\syntypes^\circ}
\newcommand{\synatoms}{{\mathcal A}}
\newcommand{\synprod}{\pmb{\times}}
\newcommand{\synarrow}{\pmb{\rightarrow}}
\newcommand{\synneg}{\pmb{\neg}}
\newcommand{\synvee}{\pmb{\vee}}
\newcommand{\synwedge}{\pmb{\wedge}}
\newcommand{\syndiff}{\pmb{\backslash}}
\newcommand{\syncap}{\operatornamewithlimits{\pmb{\bigwedge}}}
\newcommand{\syncup}{\operatornamewithlimits{\pmb{\bigvee}}}
\newcommand{\socle}[1]{\beth(#1)}
\newcommand{\appl}{\bullet}
\newcommand{\synch}[2]{\textit{ch}^{#1}\!(#2)}
\newcommand{\synchan}[1]{\textit{ch}(#1)}
\newcommand{\synchanK}{\textit{ch}}
\newcommand{\synchanout}[1]{\textit{ch}^{\!\textbf{--}\!\!}(#1)}
\newcommand{\synchanin}[1]{\textit{ch}^+(#1)}

%blackboardbold types
\def\bbbone{{\mathchoice {\rm 1\mskip-4mu l} {\rm 1\mskip-4.5mu l}
          {\rm 1\mskip-4.5mu l} {\rm 1\mskip-5mu l}}}
\newcommand{\synnatone}{\bbbone}
\newcommand{\synnatk}{\Bbbk}
\newcommand{\synnat}[1]{\mathbb{#1}}

% Les motifs
\newcommand{\pator}[2]{#1 \pmb{|} #2}
\newcommand{\patand}[2]{#1 \pmb{\wedge} #2}
%\newcommand{\patleft}[1]{\pmb{(}#1\pmb{,\_)}}
%\newcommand{\patright}[1]{\pmb{(\_,}#1\pmb{)}}
\newcommand{\patcst}[2]{\pmb{(}#1\pmb{:=}#2\pmb{)}}
\newcommand{\patpair}[2]{\pmb{(}#1\pmb{,}#2\pmb{)}}

% Filtrage
\newcommand{\erreur}{\Omega}
\newcommand{\accept}[1]{\pmb{\lbag} #1 \pmb{\rbag}}
\newcommand{\acceptd}[1]{\pmb{\lfloor} #1 \pmb{\rfloor}}
\newcommand{\semaccept}[1]{\Lbag #1 \Rbag}
\newcommand{\filter}[2]{({#1}/{#2})}%{ ({#1} \Downarrow {#2}) }
\newcommand{\semfilter}[2]{ ({#1} \downarrow {#2}) }


% Symboles du langage
%\newcommand{\myarrow}{\Pisymbol{cmt}{61}\Pisymbol{pcr}{62}}%
\newcommand{\myarrow}{\!\!\boldsymbol{\Rightarrow}\!\!} %\texttt{\char61\char62\relax}}
\newcommand{\mybar}{\texttt{|}}%
\newcommand{\motifs}{\mathbb{P}}
\newcommand{\vars}{\mathbb{V}}
\newcommand{\consts}{\mathbb{C}}
\newcommand{\exprs}{{\mathbb E}}
\newcommand{\expr}{{e}}
\newcommand{\pat}{{\tt p}}
\newcommand{\const}{{n}}
\newcommand{\op}{{o}}
\newcommand{\ops}{{\mathbb O}}
\newcommand{\valeurs}{{\cal V}}
\newcommand{\values}{\valeurs}
\newcommand{\blpar}{\boldsymbol{(}}
\newcommand{\brpar}{\boldsymbol{)}}
\newcommand{\match}[5]{\mathtt{match~} #1 \mathtt{~with~} #2 \myarrow #3 \mybar
  #4 \myarrow #5}
\newcommand{\typecase}[5]{\mathtt{typecase~} #1 \mathtt{~with~} (#2 :
  #3) \myarrow #4 \mybar (#2 : \synneg #3) \myarrow #5}
\newcommand{\typec}[5]{\texttt{(} #3=#1\pmb{\in}#2\texttt{)}\pmb{?}#4\texttt{{:}}#5}
\newcommand{\genabstraction}[3]{\abstr{f}{#1}{#2}{#3}}
%\newcommand{\abstr}[4]{\boldsymbol{\mu} #1^{\blpar #2 \brpar}\blpar #3 \brpar \boldsymbol{.}#4}
\newcommand{\stdmatch}{\match{\expr}{p_1}{\expr_1}{p_2}{\expr_2}}
\newcommand{\exprpair}[2]{\blpar #1 \boldsymbol{,} #2 \brpar}
\newcommand{\exprop}[2]{#1 \blpar #2 \brpar}

% S�mantique op�rationnelle
%\newcommand{\wrong}{\mbox{\bf wrong}}
\newcommand{\wrong}{\Omega}
%\newcommand{\reject}{\mbox{\bf reject}}
%\newcommand{\apply}{{\tt apply}}


% Enregistrements
\newcommand{\domdef}[1]{\text{Def}(#1)}
\newcommand{\labels}{{\cal L}}
\newcommand{\urecord}{{\text{\bf record}}}
\newcommand{\trec}[2]{\pmb{[} #1 : #2 \pmb{]}}
\newcommand{\tsupp}[1]{\text{\bf S}(#1)}
%\newcommand{\supp}[1]{\text{Supp}(#1)}
\newcommand{\patrec}[2]{\{ #1 : #2\}}
\newcommand{\Iff}{\Longleftrightarrow}
\newcommand{\reff}{\mathtt{ref}\,}
\newcommand{\lazy}{\mathtt{lazy}\,}


% Th�or�mes ...
%\newtheorem{theorem}{Theorem}[section]
%\newtheorem{lemma}[theorem]{Lemma}
%\newtheorem{proposition}[theorem]{Proposition}
%\newtheorem{corollary}[theorem]{Corollary}
%\newtheorem{definition}[theorem]{Definition}
%\newtheorem{condition}{Condition}
%\newenvironment{definition}{\begin{defn}}{\qed\end{defn}}
%\newtheorem{definition}[theorem]{Definition}
%\newtheorem{property}[theorem]{Property}
%\newtheorem{convention}[theorem]{Convention}
%\newtheorem{remark}[theorem]{Remark}
%\renewenvironment{proof}{{\sl Proof}:}{\qed}
